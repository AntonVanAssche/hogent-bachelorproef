%%=============================================================================
%% Methodologie
%%=============================================================================

\chapter{\IfLanguageName{dutch}{Methodologie}{Methodology}}%
\label{ch:methodologie}

%% TODO: In dit hoofstuk geef je een korte toelichting over hoe je te werk bent
%% gegaan. Verdeel je onderzoek in grote fasen, en licht in elke fase toe wat
%% de doelstelling was, welke deliverables daar uit gekomen zijn, en welke
%% onderzoeksmethoden je daarbij toegepast hebt. Verantwoord waarom je
%% op deze manier te werk gegaan bent.
%% 
%% Voorbeelden van zulke fasen zijn: literatuurstudie, opstellen van een
%% requirements-analyse, opstellen long-list (bij vergelijkende studie),
%% selectie van geschikte tools (bij vergelijkende studie, "short-list"),
%% opzetten testopstelling/PoC, uitvoeren testen en verzamelen
%% van resultaten, analyse van resultaten, ...
%%
%% !!!!! LET OP !!!!!
%%
%% Het is uitdrukkelijk NIET de bedoeling dat je het grootste deel van de corpus
%% van je bachelorproef in dit hoofstuk verwerkt! Dit hoofdstuk is eerder een
%% kort overzicht van je plan van aanpak.
%%
%% Maak voor elke fase (behalve het literatuuronderzoek) een NIEUW HOOFDSTUK aan
%% en geef het een gepaste titel.

\section{Fase 1: Stand van zaken}
\label{sec:fase-1-stand-van-zaken}

Dit onderdeel belicht de actuele benadering van configuration management, met een specifieke focus op Infrastructure as Code (IaC).
Naast een definitie van IaC worden de essentiële componenten ervan besproken, evenals de voordelen die het met zich meebrengt.
Bovendien worden tools genoemd die kunnen worden ingezet om de configuratie van een systeem te identificeren en later om te zetten naar een IaC-omgeving.

Daarnaast zal er ook aandacht aandacht besteed aan asset management en hoe dit kan helpen bij de overgang van een niet-IaC-omgeving naar een IaC-omgeving.
Hoewel asset management minder gericht is op de configuratie van individuele machines, biedt het wel inzicht in de infrastructuur en de assets die beheerd moeten worden.
Op basis van deze informatie kunnen vervolgens de initiële configuraties worden vastgelegd in code,

\section{Face 2: Risico analyse}
\label{fase-2-risico-analyse}

Dit gedeelte richt zich op de risicoanalyse die nodig is om de eigenschappen van een Linux-systeem te identificeren die cruciaal zijn voor opname in een configuratie-inventaris.
Bij de overgang naar een Infrastructure as Code (IaC)-omgeving is het essentieel om een grondige analyse uit te voeren om de belangrijkste kenmerken van een Linux-machine te bepalen die moeten worden vastgelegd in code.Met andere woorden, we gaan de identiteit van een Linux-machine blootleggen en ontdekken welke configuratie-elementen van vitaal belang zijn.

De resultaten van deze fase zullen later worden gebruikt in fase 3, de Proof of Concept.

\section{Face 3: Proof of concept}
\label{face-3-proof-of-concept}

In de Proof of Concept (PoC) fase zal een Bash-script worden ontwikkeld met als doel de configuratie van Debian Linux machines bloot te leggen.
Dit script zal worden ontworpen om op vijf servers te worden uitgevoerd, waarbij elke server zijn eigen specifieke rol vervult binnen de omgeving.

Het Bash-script zal zorgvuldig worden samengesteld om relevante configuratie-informatie te verzamelen van elke Debian Linux-machine.
Hierbij wordt gedacht aan informatie zoals specieke eigenscahppen van het besturingssysteem, alsook de configuratie van ge\"installeerde software en services.
Door deze informatie te verzamelen, kunnen we een gedetailleerd beeld krijgen van de configuratiestatus van elke server.

De vijf servers zullen elk een specifieke taak binnen de omgeving vervullen, wat betekent dat het Bash-script zal worden uitgevoerd op servers met verschillende configuraties en instellingen.
Dit stelt ons in staat om de veelzijdigheid en robuustheid van het script te testen in verschillende scenario's.

Het uiteindelijke doel van deze PoC-fase is om te valideren of het ontwikkelde Bash-script effectief in staat is om de configuratie van Debian Linux-machines te identificeren en dit op een eenoudige manier te rapporteren.

\section{Fase 4: Conclusie}
\label{fas-4-conclusie}
