%%=============================================================================
%% Methodologie
%%=============================================================================

\chapter{\IfLanguageName{dutch}{Methodologie}{Methodology}}%
\label{ch:methodologie}

%% TODO: In dit hoofstuk geef je een korte toelichting over hoe je te werk bent
%% gegaan. Verdeel je onderzoek in grote fasen, en licht in elke fase toe wat
%% de doelstelling was, welke deliverables daar uit gekomen zijn, en welke
%% onderzoeksmethoden je daarbij toegepast hebt. Verantwoord waarom je
%% op deze manier te werk gegaan bent.
%% 
%% Voorbeelden van zulke fasen zijn: literatuurstudie, opstellen van een
%% requirements-analyse, opstellen long-list (bij vergelijkende studie),
%% selectie van geschikte tools (bij vergelijkende studie, "short-list"),
%% opzetten testopstelling/PoC, uitvoeren testen en verzamelen
%% van resultaten, analyse van resultaten, ...
%%
%% !!!!! LET OP !!!!!
%%
%% Het is uitdrukkelijk NIET de bedoeling dat je het grootste deel van de corpus
%% van je bachelorproef in dit hoofstuk verwerkt! Dit hoofdstuk is eerder een
%% kort overzicht van je plan van aanpak.
%%
%% Maak voor elke fase (behalve het literatuuronderzoek) een NIEUW HOOFDSTUK aan
%% en geef het een gepaste titel.

\section{Stand van zaken}
\label{sec:stand-van-zaken}
Dit onderdeel belicht de actuele benadering van configuration management, met een specifieke focus op Infrastructure as Code (IaC).
Naast een definitie van IaC worden de essenti\"ele componenten ervan besproken, evenals de voordelen die het met zich meebrengt.
Bovendien worden tools genoemd die kunnen worden ingezet om de configuratie van een systeem te identificeren en later om te zetten naar een IaC-omgeving.

Daarnaast zal er ook aandacht worden besteed aan asset management en hoe dit kan helpen bij de overgang van een niet-IaC-omgeving naar een IaC-omgeving.
Hoewel asset management minder gericht is op de configuratie van individuele machines, biedt het wel inzicht in de infrastructuur en de assets die beheerd moeten worden.
Op basis van deze informatie kunnen vervolgens de initi\"ele configuraties worden vastgelegd in code,

\section{Linux Server Concepten}
\label{sec:linux-server-concepten}
Dit hoofdstuk richt zich op de fundamentele concepten van Linux-servers.
Het doel is om een dieper begrip te bieden van de verschillende componenten en functies van een Linux-systeem, evenals de relevante ondersteunende tools.
Door deze basisbegrippen te verkennen, kunnen we effectiever bepalen welke informatie moet worden opgenomen in onze configuratie-inventaris.

\section{Computernetwerk Concepten}
\label{sec:computernetwerk-concepten}
Dit hoofdstuk behandelt de essenti\"ele concepten van computernetwerken in relatie tot servers.
We zullen verkennen hoe netwerkconnectiviteit een integraal onderdeel is van de functionaliteit van servers en hoe verschillende concepten en technologie\"en worden toegepast om deze connectiviteit te realiseren.

\section{Het opstellen van een configuratie-inventaris}
\label{sec:risicoanalyse}
In dit hoofdstuk zullen we een risicoanalyse uitvoeren voor Linux-systemen.
We zullen de bevindingen uit de eerdere hoofdstukken combineren om een lijst van cruciale configuratie-eigenschappen te presenteren voor het beheer en de beveiliging van Linux-systemen.
Door gebruik te maken van literatuuronderzoek en de besproken concepten, zullen we de risico's identificeren waaraan Linux-systemen blootstaan en welke specifieke eigenschappen moeten worden opgenomen in onze configuratie-inventaris.

\section{Proof of Concept}
\label{sec:proof-of-concept}
Dit hoofdstuk omvat de praktische toepassing van onze bevindingen.
We zullen een Bash-script ontwikkelen en uitvoeren op 5 virtuele Debian-servers, elk met hun eigen unieke taken en eigenschappen.
Dit stelt ons in staat om de bruikbaarheid en effectiviteit van onze aanpak te testen in een real-world scenario.

We zullen ook de gevonden resultaten analyseren en beoordelen of de configuratie-inventaris voldoet aan de verwachtingen en of deze als basis kan dienen voor een Infrastructure as Code (IaC)-omgeving.
Dit wordt gedaan door de resultaten te vergelijken met de bevindingen uit de risicoanalyse en de configuratie-inventaris.
