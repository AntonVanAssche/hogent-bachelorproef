%%=============================================================================
%% Inleiding
%%=============================================================================

\chapter{\IfLanguageName{dutch}{Inleiding}{Introduction}}%
\label{ch:inleiding}

Om dit te bereiken, zal een script of tool worden ontwikkeld die automatisch op elke Linux-server kan worden uitgevoerd, en gekeken worden naar reeds bestaande oplossingen die we kunnen gebruiken.
Deze tool biedt vervolgens een gedetailleerde inventaris van het systeem en de configuratie van verschillende applicaties.
Het uiteindelijke doel is om deze tool uit te voeren op vijf verschillende servers met specifieke use cases.
Op basis hiervan zal worden geconcludeerd of het implementeren van deze tool nuttig is voor KMO's.

\section{\IfLanguageName{dutch}{Probleemstelling}{Problem Statement}}%
\label{sec:probleemstelling}

In het begin van 2023, op 6 januari, werd de NIS-richtlijn, wat staat voor "Network and Information Security", opgevolgd door zijn nieuwe versie, NIS2.
Deze update bracht verschillende nieuwe voorschriften met zich mee op het gebied van cyberbeveiliging, en het aantal sectoren dat aan deze richtlijnen moet voldoen, werd aanzienlijk uitgebreid.
Enkele sectoren die onder de NIS2 richtlijnen vallen zijn, maar niet beperkt tot~\autocite{NIS2Directive2022}:
\begin{itemize}
    \item Energie: elektriciteit, olie, gas, stadsverwarming en waterstof
    \item Transport: lucht, spoor, water en weg
    \item Digital infrastructuur: Telecom, DNS, TLD, datacenters, vertrouwensdiensten, clouddiensten, \ldots
    \item Digitale diensten: zoekmachines, online markten, sociale netwerken, \ldots
\end{itemize}

De lidstaten van de Europese Unie hebben tot oktober 2024~\autocite{NIS2Directive2022} de tijd om deze nieuwe richtlijnen in hun nationale wetgeving te implementeren.
Dit geeft bedrijven een beperkte periode om hun processen aan te passen aan deze nieuwe wetgeving, een uitdaging die vooral voor kleine of middelgrote ondernemingen (KMO's) problematisch kan zijn, gezien hun vaak beperkte middelen.

Een van de essenti\"ele vereisten van deze richtlijnen is dat bedrijven verplicht zijn om een ``inventory of assets'' op te stellen voor alle systemen en operaties die ze gebruiken.
Deze bachelorproef richt zich specifiek op het bieden van een oplossing voor KMO's, binnen sectoren die nu wel moeten voldoen aan deze richtlijnen, voor het inventariseren van Linux-systemen hun configuratie, waarbij de focus ligt op het vergemakkelijken van het naleven van deze nieuwe regelgeving.

\section{\IfLanguageName{dutch}{Onderzoeksvraag}{Research question}}%
\label{sec:onderzoeksvraag}

Kan men met behulp van een script, geschreven in Bash, KMO's ondersteunen bij het inventariseren van Linux-servers en hun configuratie, om zo te voldoen aan de NIS2-richtlijnen?

\section{\IfLanguageName{dutch}{Onderzoeksdoelstelling}{Research objective}}%
\label{sec:onderzoeksdoelstelling}

Wat is het beoogde resultaat van je bachelorproef? Wat zijn de criteria voor succes? Beschrijf die zo concreet mogelijk. Gaat het bv.\ om een proof-of-concept, een prototype, een verslag met aanbevelingen, een vergelijkende studie, enz.

Het resultaat van deze bachelorproef is een proof-of-concept, die KMO's ondersteunt in de vorm van een voorbeeldscript dat kan worden uitgevoerd op elke Linux-server, om zo te voldoen aan de nieuwe NIS2-richtlijnen.
Waar een Bash script zal worden uitgevoerd op elke server, die vervolgens een inventaris van het systeem en de configuratie van verschillende applicaties zal genereren.

\section{\IfLanguageName{dutch}{Opzet van deze bachelorproef}{Structure of this bachelor thesis}}%
\label{sec:opzet-bachelorproef}

% Het is gebruikelijk aan het einde van de inleiding een overzicht te
% geven van de opbouw van de rest van de tekst. Deze sectie bevat al een aanzet
% die je kan aanvullen/aanpassen in functie van je eigen tekst.

De rest van deze bachelorproef is als volgt opgebouwd:

In Hoofdstuk~\ref{ch:stand-van-zaken} wordt een overzicht gegeven van de stand van zaken binnen het onderzoeksdomein, op basis van een literatuurstudie.

In Hoofdstuk~\ref{ch:methodologie} wordt de methodologie toegelicht en worden de gebruikte onderzoekstechnieken besproken om een antwoord te kunnen formuleren op de onderzoeksvragen.

% TODO: Vul hier aan voor je eigen hoofstukken, één of twee zinnen per hoofdstuk

In Hoofdstuk~\ref{ch:conclusie}, tenslotte, wordt de conclusie gegeven en een antwoord geformuleerd op de onderzoeksvragen. Daarbij wordt ook een aanzet gegeven voor toekomstig onderzoek binnen dit domein.
