%%=============================================================================
%% Inleiding
%%=============================================================================

\chapter{\IfLanguageName{dutch}{Inleiding}{Introduction}}%
\label{ch:inleiding}

\section{\IfLanguageName{dutch}{Probleemstelling}{Problem Statement}}%
\label{sec:probleemstelling}

In het begin van 2023, op 6 januari, werd de NIS-richtlijn, wat staat voor "Network and Information Security", opgevolgd door zijn nieuwe versie, NIS2.
Deze update bracht verschillende nieuwe voorschriften met zich mee op het gebied van cyberbeveiliging, en het aantal sectoren dat aan deze richtlijnen moet voldoen, werd aanzienlijk uitgebreid.
Enkele sectoren die onder de NIS2 richtlijnen vallen zijn, maar niet beperkt tot~\autocite{NIS2Directive2022}:
\begin{itemize}
    \item Energie: elektriciteit, olie, gas, stadsverwarming en waterstof
    \item Transport: lucht, spoor, water en weg
    \item Digital infrastructuur: Telecom, DNS, TLD, datacenters, vertrouwensdiensten, clouddiensten, \ldots
    \item Digitale diensten: zoekmachines, online markten, sociale netwerken, \ldots
\end{itemize}

De lidstaten van de Europese Unie hebben tot oktober 2024~\autocite{NIS2Directive2022} de tijd om deze nieuwe richtlijnen in hun nationale wetgeving te implementeren.
Dit geeft bedrijven een beperkte periode om hun processen aan te passen aan deze nieuwe wetgeving, een uitdaging die vooral voor kleine of middelgrote ondernemingen (KMO's) problematisch kan zijn, gezien hun vaak beperkte middelen.

Een van de essenti\"ele vereisten van deze richtlijnen is dat bedrijven verplicht zijn om een ``inventory of assets'' op te stellen voor alle systemen en operaties die ze gebruiken.
Deze bachelorproef richt zich specifiek op het bieden van een oplossing voor KMO's, binnen sectoren die nu wel moeten voldoen aan deze richtlijnen, voor het inventariseren van Linux-systemen hun configuratie, waarbij de focus ligt op het vergemakkelijken van het naleven van deze nieuwe regelgeving.

Een manier die op dit moment veel gebruikt wordt om een inventaris van systemen en de configuratie ervan is het gebruik maken van Infrastructure as Code, ook wel IaC genoemd.
Het is een concept dat de laatste jaren steeds meer aan populariteit wint. Voornamelijk doordat het beheerders van computernetwerken in staat stelt om hun configuratie en infrastructuur vast te leggen in code, in plaats van handmatig te configureren.
Binnen deze code kunnen veelvoorkomende, complexe taken automatisch worden uitgevoerd, en dit alles op een geteste en foutloze manier~\autocite{chef-what-is-iac}.

\section{\IfLanguageName{dutch}{Onderzoeksvraag}{Research question}}%
\label{sec:onderzoeksvraag}

Wees zo concreet mogelijk bij het formuleren van je onderzoeksvraag. Een onderzoeksvraag is trouwens iets waar nog niemand op dit moment een antwoord heeft (voor zover je kan nagaan). Het opzoeken van bestaande informatie (bv. ``welke tools bestaan er voor deze toepassing?'') is dus geen onderzoeksvraag. Je kan de onderzoeksvraag verder specifiëren in deelvragen. Bv.~als je onderzoek gaat over performantiemetingen, dan 

\section{\IfLanguageName{dutch}{Onderzoeksdoelstelling}{Research objective}}%
\label{sec:onderzoeksdoelstelling}

Wat is het beoogde resultaat van je bachelorproef? Wat zijn de criteria voor succes? Beschrijf die zo concreet mogelijk. Gaat het bv.\ om een proof-of-concept, een prototype, een verslag met aanbevelingen, een vergelijkende studie, enz.

\section{\IfLanguageName{dutch}{Opzet van deze bachelorproef}{Structure of this bachelor thesis}}%
\label{sec:opzet-bachelorproef}

% Het is gebruikelijk aan het einde van de inleiding een overzicht te
% geven van de opbouw van de rest van de tekst. Deze sectie bevat al een aanzet
% die je kan aanvullen/aanpassen in functie van je eigen tekst.

De rest van deze bachelorproef is als volgt opgebouwd:

In Hoofdstuk~\ref{ch:stand-van-zaken} wordt een overzicht gegeven van de stand van zaken binnen het onderzoeksdomein, op basis van een literatuurstudie.

In Hoofdstuk~\ref{ch:methodologie} wordt de methodologie toegelicht en worden de gebruikte onderzoekstechnieken besproken om een antwoord te kunnen formuleren op de onderzoeksvragen.

% TODO: Vul hier aan voor je eigen hoofstukken, één of twee zinnen per hoofdstuk

In Hoofdstuk~\ref{ch:conclusie}, tenslotte, wordt de conclusie gegeven en een antwoord geformuleerd op de onderzoeksvragen. Daarbij wordt ook een aanzet gegeven voor toekomstig onderzoek binnen dit domein.
