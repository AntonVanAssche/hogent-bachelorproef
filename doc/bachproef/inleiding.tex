%%=============================================================================
%% Inleiding
%%=============================================================================

\chapter{\IfLanguageName{dutch}{Inleiding}{Introduction}}%
\label{ch:inleiding}

\section{\IfLanguageName{dutch}{Probleemstelling}{Problem Statement}}%
\label{sec:probleemstelling}

In het begin van 2023, op 6 januari, werd de NIS-richtlijn, wat staat voor "Network and Information Security", opgevolgd door zijn nieuwe versie, NIS2.
Deze update bracht verschillende nieuwe voorschriften met zich mee op het gebied van cyberbeveiliging, en het aantal sectoren dat aan deze richtlijnen moet voldoen, werd aanzienlijk uitgebreid.
Enkele sectoren die onder de NIS2 richtlijnen vallen zijn, maar niet beperkt tot~\autocite{NIS2Directive2022}:
\begin{itemize}
    \item Energie: elektriciteit, olie, gas, stadsverwarming en waterstof
    \item Transport: lucht, spoor, water en weg
    \item Digital infrastructuur: Telecom, DNS, TLD, datacenters, vertrouwensdiensten, clouddiensten, \ldots
    \item Digitale diensten: zoekmachines, online markten, sociale netwerken, \ldots
\end{itemize}

Het valt op dat de meeste sectoren die onder de NIS2-richtlijnen vallen, niet per se gericht zijn op IT.
Echter zijn bedrijven tegenwoordig in toenemende mate afhankelijk van bepaalde IT-processen die een impact hebben op hun bedrijfsvoering.
Dit maakt het noodzakelijk om ook cyberbeveiligingswetgevingen op hen toe te passen.

De lidstaten van de Europese Unie hebben tot oktober 2024~\autocite{NIS2Directive2022} de tijd om deze nieuwe richtlijnen in hun nationale wetgeving te implementeren.
Dit geeft bedrijven een beperkte periode om hun processen aan te passen aan deze nieuwe wetgeving, een uitdaging die vooral voor kleine of middelgrote ondernemingen (KMO's) problematisch kan zijn, gezien hun vaak beperkte middelen.

Een van de essenti\"ele vereisten van deze richtlijnen is dat bedrijven verplicht zijn om een ``inventory of assets'' op te stellen voor alle systemen en operaties die ze gebruiken.
Deze bachelorproef richt zich specifiek op het bieden van een oplossing voor KMO's, binnen sectoren die nu wel moeten voldoen aan deze richtlijnen, voor het inventariseren van Linux-systemen hun configuratie, waarbij de focus ligt op het vergemakkelijken van het naleven van deze nieuwe regelgeving.

Een manier die op dit moment veel gebruikt wordt om een inventaris van systemen en de configuratie ervan is het gebruik maken van Infrastructure as Code, ook wel IaC genoemd.
Het is een concept dat de laatste jaren steeds meer aan populariteit wint. Voornamelijk doordat het beheerders van computernetwerken in staat stelt om hun configuratie en infrastructuur vast te leggen in code, in plaats van handmatig te configureren.
Binnen deze code kunnen veelvoorkomende, complexe taken automatisch worden uitgevoerd, en dit alles op een geteste en foutloze manier~\autocite{chef-what-is-iac}.
Hier zullen we later verder op ingaan in hoofstuk~\ref{ch:stand-van-zaken}.

\section{\IfLanguageName{dutch}{Onderzoeksvraag}{Research question}}%
\label{sec:onderzoeksvraag}

\begin{enumerate}
    \item Hoe kunnen bedrijven, met name KMO's, zich aanpassen aan de nieuwe vereisten van de NIS2-richtlijn, met name met betrekking tot het opstellen van een ``inventory of assets'' voor hun IT-systemen?
    \item Op welke manieren kan Infrastructure as Code (IaC) worden ingezet om het inventariseren van systemen en configuraties te vergemakkelijken en te voldoen aan de eisen van de NIS2-richtlijn?
    \item Welke tools en technieken kunnen worden gebruikt om de configuratie en infrastructuur van Linux-systemen vast te leggen en te beheren?
    \item Welke configuratie-eigenschappen van Linux-systemen zijn essentieel om in kaart te brengen voor een inventaris?
\end{enumerate}

\section{\IfLanguageName{dutch}{Onderzoeksdoelstelling}{Research objective}}%
\label{sec:onderzoeksdoelstelling}

Het doel van deze bachelorproef is om een Proof of Concept te ontwikkelen die KMO's helpt bij het inventariseren van hun Linux-systemen en hun configuratie, met behulp van een Bash-script.
Het script zal rekening moeten houden met de bevindingen uit de risicoanalyse, en voorgaande hoofdstukken van deze bachelorproef.

Ook zal er een kritische evaluatie worden gemaakt van de Proof of Concept, waarbij de effectiviteit en volledigheid van de inventarisatie wordt beoordeeld.

\section{\IfLanguageName{dutch}{Opzet van deze bachelorproef}{Structure of this bachelor thesis}}%
\label{sec:opzet-bachelorproef}

% Het is gebruikelijk aan het einde van de inleiding een overzicht te
% geven van de opbouw van de rest van de tekst. Deze sectie bevat al een aanzet
% die je kan aanvullen/aanpassen in functie van je eigen tekst.

De rest van deze bachelorproef is als volgt opgebouwd:

In Hoofdstuk~\ref{ch:stand-van-zaken} wordt een overzicht gegeven van de stand van zaken binnen het onderzoeksdomein, op basis van een literatuurstudie.

In Hoofdstuk~\ref{ch:methodologie} wordt de methodologie toegelicht en worden de gebruikte onderzoekstechnieken besproken om een antwoord te kunnen formuleren op de onderzoeksvragen.

In Hoofdstuk~\ref{ch:linux-server-concepten} richten we ons op enkele fundamentele concepten van Linux-servers.

In Hoofdstuk~\ref{ch:computernetwerk-concepten} behandelt de essentiële concepten van computernetwerken in relatie tot servers.

In Hoofdstuk~\ref{ch:risicoanalyse} wordt een risicoanalyse uitgevoerd op de configuratie van Linux-servers, op basis van voorgaande hoofdstukken.

In Hoofdstuk~\ref{ch:poc} wordt een proof-of-concept uitgewerkt, waarbij de bevindingen van de risicoanalyse worden toegepast in een Bash-script.

In Hoofdstuk~\ref{ch:conclusie}, tenslotte, wordt de conclusie gegeven en een antwoord geformuleerd op de onderzoeksvragen. Daarbij wordt ook een aanzet gegeven voor toekomstig onderzoek binnen dit domein.
