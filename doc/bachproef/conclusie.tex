%%=============================================================================
%% Conclusie
%%=============================================================================

\chapter{Conclusie}%
\label{ch:conclusie}

Deze bachelorproef had als doel het verkennen van het gebruik van Infrastructure as Code (IaC) om te voldoen aan de NIS2-eis voor het bijhouden van een ``inventory of assets''.
Specifiek lag de focus op het ontwikkelen van een praktische oplossing voor het opstellen van een configuratie-inventaris op Linux-systemen te automatiseren.

Het onderzoek begon met een literatuurstudie waarin de fundamentele principes van IaC werden besproken, zoals idempotentie, automatisatie en testen, en onderzoekt tools zoals Puppetboard.
Vervolgens worden tools zoals Nmap en linPEAS-ng besproken voor het ontdekken van netwerk- en configuratie-eigenschappen van Linux-systemen.
Zo bleek dat Nmap reeds gebruikt wordt om netwerken in kaart te brengen, en dus ons kan helpen bij het verzamelen van netwerkconfiguratie, zoals het detecteren van open poorten binnen het netwerk als ook hosts te ontdekken.
LinPEAS-ng kan dan weer helpen bij het identificeren van gevoelige configuratie-eigenschappen zoals rechten van bestanden en gebruikers en is meer gericht op het vinden van kwetsbaarheden in de configuratie van een systeem.
Ook werd er gekeken naar huidige asset management-tools zoals Snipe-IT, Lansweeper en NetBox, die van belang zijn voor het inventariseren van hardware, software en netwerkgerelateerde zaken.

Naast een literatuurstudie werd er ook een risicoanalyse uitgevoerd om de belangrijkste configuratie-eigenschappen van Linux-systemen te identificeren die moeten worden opgenomen in de configuratie-inventaris.
Hierbij lag de focus op factoren die de identiteit van een Linux-systeem bepalen, zoals het besturingssysteem, netwerkinterfaces, ge\"installeerde software en beveiligingsaspecten.
Deze resultaten werden vervolgens gebruikt om een Proof of Concept te ontwikkelen, waarbij een Bash-script genaamd ConfiScan werd ontwikkeld, specifiek ontworpen voor Debian.
Het script verzamelt configuratie-eigenschappen en slaat deze op in CSV- en tekstbestanden.
Het script werd getest op vijf virtuele machines met Vagrant en Ansible, waarbij enkele beperkingen en mogelijke oplossingen werden ge\"identificeerd.

Ondanks enkele beperkingen biedt het script een solide basis voor verdere ontwikkeling en uitbreiding.
Toekomstige verbeteringen kunnen het volgen van soft- en hardlinks en het vastleggen van SELinux- of AppArmor-configuraties omvatten.
Het script kan verder worden ge\"integreerd in bestaande systeembeheerprocessen en aangepast voor andere besturingssystemen en cloudomgevingen, wat de bruikbaarheid vergroot.
Dit onderzoek draagt bij aan systeembeheer en biedt een praktische oplossing voor geautomatiseerde configuratie-inventarisatie op Linux-servers.
