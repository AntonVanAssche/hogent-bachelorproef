%%=============================================================================
%% Methodologie
%%=============================================================================

\chapter{\IfLanguageName{dutch}{Het opstellen van een configuratie-inventaris}{The creation of a configuration inventory}}
\label{ch:risicoanalyse}

\section{Inleiding}
\label{risico_inleiding}

De overgang van een traditionele omgeving naar een Infrastructure as Code (IaC) omgeving vereist een grondige kennis van de configuratie van Linux-systemen.
Een cruciale stap in dit proces is het opstellen van een gedetailleerde configuratie-inventaris die alle essenti\"ele eigenschappen van het systeem omvat.
Deze inventaris vormt de basis voor het automatiseren van de infrastructuur met behulp van IaC-tools zoals Ansible, Terraform of Puppet.
In dit hoofdstuk zullen we de belangrijkste elementen bespreken die moeten worden opgenomen in een configuratie-inventaris van Linux-systemen voor een succesvolle overgang naar een IaC-omge-\ ving, waarbij we ook de relevante beveiligingsaspecten benadrukken.

\section{Het besturingssysteem}
\label{risico_besturingssysteem}

Een eerste stap bij het opstellen van een configuratie-inventaris is het verzamelen van basisinformatie over het Linux-systeem.
Dit omvat details zoals het besturingssysteem (bijvoorbeeld Debian, Ubuntu, CentOS), de versie van de kernel, en de hardware-specificaties zoals CPU-model, geheugen en opslagcapaciteit.
Het is essentieel om deze informatie vast te leggen om een goed beeld te krijgen van de huidige configuratie van het systeem.

Uit~\ref{linux_linux_besturingssysteem}, bleek dat de kernel een cruciaal onderdeel is van het besturingssysteem en een directe impact heeft op de beveiliging en prestaties van het systeem.
Hierdoor is het belangrijk om kernel die op dit moment gebruikt wordt, te documenteren in de configuratie-inventaris, alsook alle kernel-modules die momenteel geladen zijn.
Ook de versie van het besturingssysteen is van belang, omdat verschillende versies van Linux-distributies verschillende pakketten en configuraties kunnen bevatten.

\subsection{Gebruikers- en groepsbeheer}
\label{risico_gebruikers_groepen}

Het beheren van gebruikers en groepen vormt een essentieel aspect van systeemconfiguratie.
Dit is van cruciaal belang omdat gebruikers vaak worden aangemaakt om specifieke taken uit te voeren, zoals bijvoorbeeld de \texttt{www-data} gebruiker die doorgaans wordt gebruikt voor het beheer van de Apache webserver, zoals besproken in~\ref{linux_gebruikers_groepen}.
In de configuratie-inventaris moeten alle gebruikersaccounts en groepen op het systeem worden opgenomen, inclusief hun respectievelijke rechten en permissies.
Dit omvat ook informatie over speciale gebruikersaccounts zoals de rootgebruiker, systeemgebruikers en de geconfigureerde \texttt{sudo}-rechten van elke gebruiker.

Door deze gegevens op te nemen in de configuratie-inventaris, kan men later controleren welke gebruikers en groepen daadwerkelijk op het systeem aanwezig moeten zijn om de vereiste taken uit te voeren, en welke niet.
Door de geregistreerde \texttt{sudo}-regels te analyseren, kan men bepalen of een bepaalde gebruiker over de juiste machtigingen beschikt, of dat het beter is om het aantal permissies te beperken.

\subsection{Bestandssysteem en partities}
\label{risico_bestandsysteem_partities}

Bij het installeren of updaten van een systeem is het essentieel om aandacht te besteden aan de opslagmedia, en hun configuratie.
Dit omvat onder andere het kiezen van het juiste bestandssysteem, het configureren van partities en het bepalen van de mountpoints~\autocite{wirzenius1993linux}.
Een grondig begrip van de bestaande bestandssystemen, partities en hun mountpoints is daarom van groot belang.

Dit belang wordt ook benadrukt in het boek "IT Inventory and Resource Management with OCS Inventory" van Barzan ``Tony'' Antal~\autocite{Antal2010}.
De auteur benadrukt welke systeemeigenschappen cruciaal zijn om op te nemen in een configura-\ tie-inventaris.
Hoewel het boek niet specifiek gericht is op Linux-systemen, zijn de principes die worden besproken van toepassing op Linux-omgevingen.

Het is eveneens een optie om back-ups te maken van essenti\"ele bestanden en directories, zoals de \texttt{/etc/} directory waar belangrijke configuratiebestanden worden bewaard.
Andere directories die het waard zijn om te monitoren, zijn onder meer \texttt{/bin/}, \texttt{/sbin/}, \texttt{/usr/bin/} en \texttt{/usr/sbin/}, aangezien deze cruciale systeemprogramma's bevatten~\autocite{linuxfoundation-filesystem}.
Een volgende stap zou zijn om een hash te berekenen van deze bestanden en deze hash vervolgens op te nemen in de configuratie-inventaris.
Dit kan worden gedaan met behulp van een tool zoals \texttt{sha256sum} om de hash van een bestand te genereren.

Het toepassen van hashberekeningen draagt bij aan een verhoogde integriteit van de data, aangezien het ons de mogelijkheid biedt om te controleren of de bestanden zijn gewijzigd.
Immers, de hash van een bestand verandert wanneer de inhoud van het bestand wijzigt.
Bovendien is het belangrijk om te vermelden dat hashfuncties one-way zijn, wat betekent dat het niet mogelijk is om de originele inhoud van het bestand te herstellen aan de hand van de hash~\autocite{herrero2021file}.

Een praktisch voorbeeld van het gebruik van \texttt{sha256sum} om de hash van bestanden te berekenen, wordt gedemonstreerd in listing~\ref{lst:sha256sum}.
Hier kan men zien dat er drie directories zijn: \texttt{a}, \texttt{b} en \texttt{c}, met in \texttt{a} en \texttt{c} een bestand genaamd \texttt{a.txt}, en in \texttt{b} een bestand genaamd \texttt{b.txt}.
Het resultaat van \texttt{sha256sum} op de bestanden toont dat \texttt{a.txt} in \texttt{a} en \texttt{c} dezelfde hashwaarde heeft, wat aangeeft dat de bestanden identiek zijn, \texttt{b.txt} in \texttt{b} heeft een andere hashwaarde, wat aangeeft dat het bestand verschilt van de andere twee.
Dit geeft aan dat de hashwaarden onafhankelijk zijn van de bestandslocatie, en dat ze een unieke identificatie vormen voor de inhoud van het bestand.

\begin{listing}
  \begin{minted}[linenos,tabsize=4,breaklines]{console}
[anton@fedorable ~]$ find {a,b,c}/ -type f -exec sha256sum {} \;
87428fc522803d31065e7bce3cf03fe475096631e5e07bbd7a0fde60c4cf25c7  a/a.txt
0263829989b6fd954f72baaf2fc64bc2e2f01d692d4de72986ea808f6e99813f  b/b.txt
87428fc522803d31065e7bce3cf03fe475096631e5e07bbd7a0fde60c4cf25c7  c/a.txt
  \end{minted}
  \caption{Uitvoer van het \texttt{find}-commando om bestanden te vinden en de hash ervan te berekenen met \texttt{sha256sum}.}
  \label{lst:sha256sum}
\end{listing}

\section{Software-installaties en -configuraties}
\label{risico_software}

De nodige software-installaties op een systeem kunnen vari\"eren afhankelijk van de specifieke vereisten van de toepassingen die op het systeem worden uitgevoerd.
Dit kan verschillende cruciale aspecten omvatten, zoals de ge\"installeerde software en de bijbehorende pakketten, evenals de configuratie van deze software.
Een grondige evaluatie en documentatie van de ge\"installeerde software en de bijbehorende pakketten, alsook hun configuratie, is essentieel voor een succesvolle overgang naar een IaC-omgeving.

\subsection{Ge{\"i}nstalleerde applicaties en bronnen}
\label{risico_software_installaties_bronnen}

Een grondig begrip van de ge\"installeerde software en de bijbehorende pakketten is essentieel voor een succesvolle overgang naar een IaC-omgeving en is ook van cruciaal belang op het gebied van beveiliging.
Het vastleggen van deze informatie in de configuratie-inventaris is noodzakelijk voor een effectieve beveiligingsstrategie.

De configuratie-inventaris moet een gedetailleerde lijst bevatten van alle ge\"installeerde softwarepakketten, inclusief hun versies en afhankelijkheden, evenals informatie over de bronnen waaruit de software is ge\"installeerd.
Dit omvat zowel systeemniveau-software als gebruikersniveau-toepassingen.
Het verzamelen van deze informatie kan worden uitgevoerd met behulp van package managers zoals \texttt{dpkg} of \texttt{rpm} om een lijst van ge\"installeerde pakketten te genereren.
Ook moeten de geconfigureerde repositories worden opgenomen in de configuratie-inventaris, aangezien deze de bron zijn voor vele software-installaties.

Een treffend voorbeeld dat illustreert waarom dit zo belangrijk is, betreft CVE-2024-3094~\autocite{nist-CVE-2024-3094}.
Deze backdoor werd ontdekt in maart 2024, in de veelgebruikte software genaamd \texttt{xz-utils}, een standaard compressie- en decompressietool, dat deel uitmaakt van de \texttt{liblzma}-biblio-\ theek.
Deze bibliotheek wordt vaak gebruikt op Linux-systemen door andere programma's.
Het werd opgemerkt nadat Andres Freund, een softwareontwikkelaar, verdachte code had gevonden in de broncode van het pakket.
Echter bleek de backdoor alleen aanwezig te zijn in versies 5.6.0 en 5.6.1 van de software, die werden uitgebracht in februari 2024~\autocite{lins2024critical}.
Op dat moment waren de ge\"infecteerde versies beschikbaar in de repositories van verschillende Linux-distributies, waaronder Debian (testing en unstable), Alpine Linux (Edge) en Fedora (40, 41 en Rawhide).

\subsection{Configuratie van ge{\"i}nstalleerde software}
\label{risico_software_configuratie}

Met enkel en alleen de ge\"installeerde software te documenteren, is het echter niet voldoende, aangezien het later de bedoeling is om aan de hand van dit inventaris een IaC-omgeving op te zetten.
Voor een volledige overgang naar een IaC-omgeving is het ook noodzakelijk om de huidige configuratie van de software vast te leggen, zodat men deze later kan omvormen naar code voor de gekozen IaC-tool.

Op Linux-systemen worden configuratiebestanden vaak bewaard in de \texttt{/etc/} directory, afgeleid van ``et cetera''~\autocite{linuxfoundation-filesystem}.
De bestanden in deze directory bevatten vaak de configuratie-opties voor verschillende softwarepakketten, en het is mogelijk om deze bestanden op te nemen in een configuratie-inventaris.

\subsection{Taakbeheer}
\label{risico_taakbeheer}

Systemd werd reeds besproken in~\ref{linux_systemd} als de standaard init-systeem op de meeste Linux-distributies.
Het is een manier om services, en processen, te starten en stoppen op een bepaald tijdstip, of bij een bepaalde gebeurtenis.
Zo kunnen we bijvoorbeeld een webserver automatisch laten starten bij het opstarten van het systeem.
Dit doen we met het \texttt{systemctl} commando gevolgd door de \texttt{enable} optie, zoals te zien in listing~\ref{lst:systemd_enable}.

\begin{listing}
  \begin{minted}[linenos,tabsize=4,breaklines]{console}
[root@fedorable ~]$ systemctl enable sshd
Created symlink /etc/systemd/system/multi-user.target.wants/sshd.service → /usr/lib/systemd/system/sshd.service.
  \end{minted}
  \caption{Uitvoer van het \texttt{systemctl}-commando om een service automatisch te laten starten bij het opstarten van het systeem.}
  \label{lst:systemd_enable}
\end{listing}

Doordat systemd units vaak worden gebruikt om services te beheren, is het van belang om deze units op te nemen in de configuratie-inventaris.
Netzoals het bijhouden van een lijst van services die automatisch worden gestart bij het opstarten van het systeem, dit kan worden bereikt door \texttt{systemctl} te gebruiken met de \texttt{list-unit-files} optie, zoals te zien in listing~\ref{lst:systemd_list_unit_files}.

\begin{listing}
  \begin{minted}[linenos,tabsize=4,breaklines]{console}
[root@fedorable ~]$ systemctl list-unit-files --state=enabled
UNIT FILE                          STATE   PRESET
cups.path                          enabled enabled
abrt-journal-core.service          enabled enabled
abrt-oops.service                  enabled enabled
abrt-vmcore.service                enabled enabled
abrt-xorg.service                  enabled enabled
[...]
  \end{minted}
  \caption{Uitvoer van het \texttt{systemctl}-commando om een lijst van services te tonen die automatisch worden gestart bij het opstarten van het systeem.}
  \label{lst:systemd_list_unit_files}
\end{listing}

Verder werd in~\ref{linux_cronjobs} besproken dat cronjobs worden gebruikt om taken op een gepland tijdstip uit te voeren.
Dit zou kunnen gebruikt worden als alternatief voor systemd timers, daarom is het ook hier belangrijk om deze cronjobs op te nemen in de configuratie-inventaris.
Dit zou men kunnen bereiken door een kopie te maken van de directories \texttt{/etc/cron.d/}, \texttt{/etc/cron.daily/}, \texttt{/etc/cron.hourly/}, \texttt{/etc/cron.monthly/} en \texttt{/etc/cron.weekly/}, alsook \texttt{/etc/crontab} die alle cronjobs bevat.

\section{Netwerkconfiguratie}
\label{risico_netwerkconfiguratie}

Vele Linux-systemen maken gebruik van netwerkverbindingen om te communiceren met andere systemen en diensten.
Dit werd reeds besproken in~\ref{ch:computernetwerk-concepten}, waarbij de basisprincipes van computernetwerken werden toegelicht, alsook enkele protocollen die worden gebruikt voor netwerkcommunicatie.
Een basis oplijsting van de netwerkconfiguratie is essentieel voor een succesvolle overgang naar een IaC-omgeving, omdat het de basis vormt voor de netwerkautomatisering.
Ook zorgt het opnemen van de netwerkconfiguratie in de configuratie-inventaris ervoor dat men een begrip heeft over welke systemen met elkaar kunnen communiceren en welke niet.

\subsection{Netwerkinterfaces}
\label{risico_netwerkinterfaces}

In~\ref{netwerk_netwerkinterface} hebben we al besproken dat netwerkinterfaces de fysieke of virtuele verbindingen zijn waardoor een Linux-systeem communiceert met andere systemen.
Elke interface heeft een IP-adres, of het nu IPv4 of IPv6 is, samen met een subnetmask en een gateway.
Deze kenmerken stellen het systeem in staat om uniek ge\"identificeerd te worden binnen het netwerk.

Het is daarom van groot belang om de netwerkinterfaces op te nemen in de confi-\ guratie-inventaris, inclusief hun configuratiegegevens zoals het IP-adres, subnetmask en gateway.
Of het nu gaat om een interface die zijn instellingen verkrijgt via DHCP of handmatig is geconfigureerd, het vastleggen van deze informatie is van cruciaal belang.

Wanneer een IP-adres wordt verkregen via een DHCP-lease, is ook informatie over de DHCP-server relevant.
Dit omvat niet alleen het IP-adres van de server, maar ook de lease-tijd en eventuele configuratieopties die door de server aan het systeem zijn verstrekt.
Zoals aangegeven door Antal in het boek ``It Inventory and Resource Management with OCS Inventory''~\autocite{Antal2010}, is het vastleggen van deze informatie essentieel voor het beheer van de netwerkconfiguratie.

\subsection{Routing en poortbeheer}
\label{risico_routing_poorten}

In hoofdstuk~\ref{ch:computernetwerk-concepten} hebben we al besproken hoe routing en poortbeheer cruciale elementen zijn binnen netwerkconfiguratie.
Poorten worden gebruikt om applicaties met andere systemen te laten communiceren, terwijl routing ervoor zorgt dat het verzonden verkeer zijn bestemming bereikt.

Het documenteren van de momenteel gebruikte poorten in de configuratie-inven-\ taris biedt inzicht in de communicatiekanalen die actief zijn.
Dit stelt beheerders in staat om een volledig overzicht te hebben van welke applicaties welke poorten gebruiken, wat van onschatbare waarde is bij het beheren en beveiligen van het netwerk.

Bij het bekijken van de routingtabel is het van essentieel belang om de huidige routingconfiguratie vast te leggen.
Dit omvat niet alleen de standaardgateway, maar ook eventuele statische routes die zijn geconfigureerd op het systeem.
Het vastleggen van deze informatie zorgt ervoor dat het verkeer effici\"ent wordt geleid naar de juiste bestemmingen, wat de algehele netwerkprestaties en betrouwbaarheid ten goede komt.

\subsection{DNS-configuratie}
\label{risico_dns}

DNS speelt een cruciale rol in het vertalen van domeinnamen naar IP-adressen, zoals besproken in~\ref{netwerk_dns}.
Een goed functionerend netwerk vereist dan ook een nauwkeurige DNS-configuratie, omdat deze ervoor zorgt dat systemen correct kunnen communiceren met andere systemen en diensten.
Daarom is het van groot belang om de DNS-configuratie vast te leggen in de configuratie-inventaris.

Naast het operationele belang is het documenteren van de DNS-configuratie ook vanuit een beveiligingsperspectief essentieel.
Door deze informatie te registreren, kunnen beheerders controleren welke DNS-servers momenteel worden gebruikt en kunnen ze beoordelen of deze veilig zijn.
Dit is van cruciaal belang om aanvallen zoals DNS-hijacking te voorkomen, waarbij een aanvaller de controle over een DNS-server overneemt om verkeer om te leiden naar kwaadaardige websites door de DNS-records op de server te wijzigen~\autocite{shaikh2020overcoming}.
Het proactief documenteren van de DNS-configuratie stelt beheerders in staat om potenti\"ele beveiligingsrisico's te identificeren en te vermijden, waardoor de algehele veiligheid van het netwerk wordt versterkt.

\section{Beveiligingsconfiguratie}
\label{risico_beveiligingsconfiguratie}

De overgang van een traditionele omgeving naar een Infrastructure as Code (IaC) omgeving brengt verschillende belangrijke aspecten met zich mee, met name op het gebied van beveiliging.
Deze overgang vereist een zorgvuldige herziening en documentatie van de beveiligingsmaatregelen om ervoor te zorgen dat de IaC-omgeving dezelfde, zo niet betere, beveiligingsniveaus biedt dan de voorafgaande traditionele omgeving.

\subsection{Firewall}
\label{risico_firewall}

Een van de pijlers van Linux-serverbeveiliging is de firewall, die inkomend en uitgaand verkeer reguleert om ongewenste toegang tot het systeem te voorkomen.
Bij de overgang naar IaC is het noodzakelijk om de configuratie van de firewall, zoals beheerd door \texttt{firewalld} of \texttt{nftables}, gedetailleerd te documenteren.
Dit omvat het defini\"eren van toegestane en geblokkeerde poorten, services en zones, evenals eventuele aangepaste regels die zijn geconfigureerd om specifieke beveiligingsvereisten te vervullen.

\subsection{SELinux}
\label{risico_selinux}

Naast de firewall is SELinux een cruciaal onderdeel van Linux-serverbeveiliging, met zijn Mandatory Access Control (MAC) model dat extra bescherming biedt bovenop de standaard Discretionary Access Control (DAC) implementatie.
Bij de overgang naar een IaC-omgeving is het van vitaal belang om de SELinux-configuratie op te nemen in de inventaris.
Dit omvat het documenteren van de SELinux-policy's, contexten en beleidsregels die van toepassing zijn op verschillende systeembronnen en processen.
Door de SELinux-configuratie te documenteren, kan het IaC-proces deze beveiligingsmaatregelen automatiseren en consistent toepassen op alle servers in de omgeving.

\subsection{AppArmor}
\label{risico_apparmor}

Hoewel SELinux de standaard is voor beveiligingsbeleid op RHEL-gebaseerde systemen, gebruiken sommige Debian-gebaseerde systemen zoals Ubuntu AppArmor als alternatief beveiligingsframework.
Bij het overgaan naar een IaC-omgeving is het belangrijk om de configuratie van AppArmor, indien van toepassing, te documenteren.
Dit omvat het vastleggen van de AppArmor-profielen en de beperkingen die ze opleggen aan de toegang van applicaties tot systeembronnen.
Hoewel AppArmor enigszins verschillend is van SELinux in zijn aanpak, is het nog steeds essentieel om de beveiligingsinstellingen ervan te begrijpen en te integreren in de IaC-workflows om een consistente beveiliging te waarborgen.

\section{Conclusie}
\label{risico_conclusie}

In dit hoofdstuk hebben we de belangrijkste elementen besproken die moeten worden opgenomen in een configuratie-inventaris van Linux-systemen voor een succesvolle overgang naar een IaC-omgeving, maar ook vanuit een beveiligingsperspectief.
Tabel~\ref{table:risico_conclusie} vat de belangrijkste elementen samen die moeten worden opgenomen in de configuratie-inventaris.

\begin{table}[!h]
    \begin{center}
        \begin{tabular}{ c c }
            \hline
                Categorie & Beschrijving \\ [0.5ex] 
            \hline
                Applicaties              & Ge\"installeerde software, bronnen, configuratie \\
                Bestandssysteem          & Bestandssystemen, partities, mountpoints \\
                Besturingssysteem        & Kernelversie, distributie, hardware-specificaties \\
                Beveiligingsconfiguratie & Firewall, SELinux, AppArmor \\
                Gebruikers en groepen    & Gebruikersaccounts, groepen, \texttt{sudo}-rechten \\
                Netwerkconfiguratie      & Netwerkinterfaces, routing, poorten, DNS \\
                Taakbeheer               & Systemd units, cronjobs \\
        \end{tabular}
    \end{center}
    \caption{Overzicht van de belangrijkste elementen voor een configuratie-inventaris van Linux-systemen.}
    \label{table:risico_conclusie}
\end{table}
