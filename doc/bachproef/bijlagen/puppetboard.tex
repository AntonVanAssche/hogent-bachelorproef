%%=============================================================================
%% Puppetboard
%%=============================================================================

\chapter{\IfLanguageName{dutch}{Puppetboard}{Puppetboard}}%
\label{ch:bijlage_puppetboard}

Afbeelding~\ref{fig:puppetboard-home} toont een voorbeeld van de startpagina van Puppetboard.
Hier krijgen we een beknopt overzicht van het aantal nodes dat door Puppet wordt beheerd en de totale hoeveelheid resources die wordt beheerd.
Ook kunnen we zien hoeveel runs zijn mislukt en hoeveel succesvol waren.
Onder de grafieken bevindt zich een lijst van servers waarop Puppet recentelijk een run heeft uitgevoerd om de gedefinieerde configuratie toe te passen.

\begin{figure}[h!]
    \includegraphics[width=\textwidth]
    {./graphics/state-of-the-art/puppetboard/puppetboard-home.png}
    \caption[Puppetboard startpagina.]{\label{fig:puppetboard-home}Voorbeeld van de Puppetboard startpagina.}
\end{figure}

Afbeelding~\ref{fig:puppetboard-inventory} toont een overzicht van alle nodes die door Puppet worden beheerd.
Alsook wanneer de laatste run was en of deze succesvol was.

\begin{figure}[h!]
    \includegraphics[width=\textwidth]
    {./graphics/state-of-the-art/puppetboard/puppetboard-hosts.png}
    \caption[Puppetboard inventaris van nodes.]{\label{fig:puppetboard-inventory}Voorbeeld van de Puppetboard waar we alle nodes van het inventaris kunnen zien.}
\end{figure}

Afbeelding~\ref{fig:puppetboard-example-4} toont een lijst van verschillende nodes die door Puppet worden beheerd.
Men kan de hostname, IP-adres, besturingssysteem, CPU-arhitectuur, kernelversie en de Puppet-agent versie zien.

\begin{figure}[h!]
    \includegraphics[width=\textwidth]
    {./graphics/state-of-the-art/puppetboard/puppetboard-inventory.png}
    \caption[Basisinformatie op Puppetboard.]{\label{fig:puppetboard-example-4}Voorbeeld van de Puppetboard waar we basis informatie over het systeem kunnen vinden.}
\end{figure}
