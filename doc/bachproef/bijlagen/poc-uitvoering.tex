%%=============================================================================
%% PoC uitvoering
%%=============================================================================

\chapter{\IfLanguageName{dutch}{Uitvoer: Proof of Concept}{Execution: Proof of Concept}}
\label{ch:bijlage_poc_uitvoering}

Deze bijlage biedt een uitgebreide beschrijving van de uitvoering van de Proof of Concept op de vijf servers.

Om het script op een reproduceerbare manier uit te voeren, is gebruik gemaakt van een Ansible playbook.
Het script heeft enkele vereisten, namelijk: \texttt{net-tools} en \texttt{dmidecode}, zoals te zien in listing~\ref{lst:bijlage-requirements}.
Het \texttt{net-tools} pakket wordt gebruikt om de configuratie van de netwerkinterfaces te verzamelen, terwijl \texttt{dmidecode} wordt gebruikt om de hardware-informatie van de servers te verzamelen.

\begin{listing}
  \begin{minted}[linenos,tabsize=4,breaklines]{yaml}
- name: Install required packages
  ansible.builtin.apt:
    name:
      - net-tools
      - dmidecode
    state: latest
    update_cache: yes
  \end{minted}
  \caption[Installatie van vereiste pakketten.]{Code verantwoordelijk voor het installeren van de vereiste pakketten}
  \label{lst:bijlage-requirements}
\end{listing}

\begin{listing}
  \begin{minted}[linenos,tabsize=4,breaklines]{yaml}
- name: Install ConfiScan
  ansible.builtin.copy:
    src: /vagrant/ansible/files/confiscan.sh
    dest: /tmp/confiscan.sh
  \end{minted}
  \caption[Kopi\"{e}ren van script naar servers.]{Code verantwoordelijk voor het kopiëren van het script naar de servers}
  \label{lst:bijlage-copy-script}
\end{listing}

Het Ansible playbook plaatst eerst het script, dat vooraf moet worden geplaatst in \texttt{src/poc/ansible/files/}, op de servers in de map \texttt{/tmp/}, zoals te zien in listing \ref{lst:bijlage-copy-script}.
Daar wordt het script vervolgens uitgevoerd met de \texttt{-t} optie, gevolgd door de paden van de configuraties die men wil opnemen in ons inventaris.
Zoals weergegeven in listing \ref{lst:bijlage-run-script}, wordt het script uitgevoerd op \texttt{srv1}.
Nadat het script op elke server is uitgevoerd, downloadt en plaatst het playbook de tarballs van elke server in de home directory van de gebruiker die het playbook uitvoert.
Ten slotte worden de tarballs uitgepakt aan het einde van het playbook.
De laatste twee stappen zijn te zien in listing \ref{lst:bijlage-tarballs}.
Listing~\ref{lst:bijlage-run-playbook} illustreert hoe men aan de hand van het playbook het script kunnen uitvoeren op elke server.

\begin{listing}
  \begin{minted}[linenos,tabsize=4,breaklines]{yaml}
- name: ConfiScan on srv1
  hosts:
    - srv1
  gather_facts: no
  tasks:
    - name: Runs script
      ansible.builtin.shell: /tmp/confiscan.sh -t /etc/bind/named.conf /etc/node_exporter/config.yaml
      args:
        chdir: /tmp/
      become: true
    \end{minted}
    \caption[Uitvoeren van script op \texttt{srv1}.]{Code verantwoordelijk voor het uitvoeren van het script op srv1}
    \label{lst:bijlage-run-script}
\end{listing}

\begin{listing}
  \begin{minted}[linenos,tabsize=4,breaklines]{yaml}
- name: Download tarballs
  hosts:
    - servers
  gather_facts: no
  tasks:
    - name: Copy tarballs
      ansible.builtin.fetch:
        src: "/tmp/{{ inventory_hostname }}-configs.tar.gz"
        dest: "{{ lookup('ansible.builtin.env', 'HOME') }}"
        owner: "{{ lookup('ansible.builtin.env', 'USER') }}"
        group: "{{ lookup('ansible.builtin.env', 'USER') }}"
    - name: Extracting tarballs
      ansible.builtin.unarchive:
        src: "{{ lookup('ansible.builtin.env', 'HOME') }}/{{ inventory_hostname }}/tmp/{{ inventory_hostname }}-configs.tar.gz"
        dest: "{{ lookup('ansible.builtin.env', 'HOME') }}/"
      delegate_to: localhost
  \end{minted}
  \caption[Downloaden en uitpakken van tarballs.]{Code verantwoordelijk voor het downloaden en uitpakken van de tarballs}
  \label{lst:bijlage-tarballs}
\end{listing}

\begin{listing}
  \begin{minted}[linenos,tabsize=4,breaklines]{console}
[vagrant@control ~]$ cd /vagrant/ansible/
[vagrant@control ~]$ ansible-playbook -i inventory.yml poc.yml
  \end{minted}
  \caption[Uitvoeren van Ansible playbook.]{Instructies voor het uitvoeren van de Ansible playbook die het script uitvoert op elke server.}
  \label{lst:bijlage-run-playbook}
\end{listing}

Echter kan men ook het script handmatig uitvoeren op elke server.
Hiervoor moet men eerst inloggen op de server en het script vanuit \texttt{/vagrant/ansible/files/} kopi\"eren naar een gewenste locatie.
In \ref{lst:bijlage-run-script-manual} wordt gedemonstreerd hoe dit handmatig kan worden gedaan op \texttt{srv1}.

Men gebruikt de \texttt{-t} optie om aan te geven dat men aan het einde een tarball wil verkrijgen in plaats van een directory genaamd \texttt{srv1-configs}.
Daarna specificeert men de gewenste configuratiebestanden die men wil opnemen in de inventaris.

Merk op dat het script moet worden uitgevoerd met root-privileges.
Dit kan worden bereikt door in te loggen als de root-gebruiker of door het gebruik van het \texttt{sudo}-commando.

\begin{listing}
  \begin{minted}[linenos,tabsize=4,breaklines]{console}
root@srv1:~# cp /vagrant/ansible/files/confiscan.sh ~
root@srv1:~# ./confiscan.sh -t \
    /etc/bind/named.conf \
    /etc/node_exporter/config.yaml
  \end{minted}
  \caption[Manueel uitvoeren van script op \texttt{srv1}.]{Instructies om het script manueel uit te voeren op \texttt{srv1}.}
  \label{lst:bijlage-run-script-manual}
\end{listing}
