%%=============================================================================
%% PoC uitvoering
%%=============================================================================

\chapter{\IfLanguageName{dutch}{Proof of Concept Uitvoering}{Proof of Concept Execution}}
\label{ch:bijlage_poc_uitvoering}

Deze bijlage biedt een uitgebreide beschrijving van de uitvoering van de Proof of Concept op de vijf servers.

Om het script op een reproduceerbare manier uit te voeren, is gebruik gemaakt van een Ansible playbook.
Het script heeft enkele vereisten, namelijk: \texttt{net-tools} en \texttt{dmidecode}, zoals te zien in listing~\ref{lst:bijlage-requirements}.
Het \texttt{net-tools} pakket wordt gebruikt om de configuratie van de netwerkinterfaces te verzamelen, terwijl \texttt{dmidecode} wordt gebruikt om de hardware-informatie van de servers te verzamelen.

\begin{listing}
  \begin{minted}[linenos,tabsize=4,breaklines]{yaml}
- name: Install required packages
  ansible.builtin.apt:
    name:
      - net-tools
      - dmidecode
    state: latest
    update_cache: yes
  \end{minted}
  \caption{Code verantwoordelijk voor het installeren van de vereiste pakketten}
  \label{lst:bijlage-requirements}
\end{listing}

\begin{listing}
  \begin{minted}[linenos,tabsize=4,breaklines]{yaml}
- name: Install ConfiScan
  ansible.builtin.copy:
    src: /vagrant/ansible/files/confiscan.sh
    dest: /tmp/confiscan.sh
  \end{minted}
  \caption{Code verantwoordelijk voor het kopiëren van het script naar de servers}
  \label{lst:bijlage-copy-script}
\end{listing}

Het Ansible playbook plaatst eerst het script, dat vooraf moet worden geplaatst in \texttt{src/poc/ansible/files/}, op de servers in de map \texttt{/tmp/}, zoals te zien in listing \ref{lst:bijlage-copy-script}.
Daar wordt het script vervolgens uitgevoerd met de \texttt{-t} optie, gevolgd door de paden van de configuraties die we willen opnemen in onze inventaris.
Zoals weergegeven in listing \ref{lst:bijlage-run-script}, wordt het script uitgevoerd op \texttt{srv1}.
Nadat het script op elke server is uitgevoerd, downloadt en plaatst het playbook de tarballs van elke server in de home directory van de gebruiker die het playbook uitvoert.
Ten slotte worden de tarballs uitgepakt aan het einde van het playbook.
De laatste twee stappen zijn te zien in listing \ref{lst:bijlage-tarballs}.

\begin{listing}
  \begin{minted}[linenos,tabsize=4,breaklines]{yaml}
- name: ConfiScan on srv1
  hosts:
    - srv1
  gather_facts: no
  tasks:
    - name: Runs script
      ansible.builtin.shell: /tmp/confiscan.sh -t /etc/bind/named.conf /etc/node_exporter/config.yaml
      args:
        chdir: /tmp/
      become: true
    \end{minted}
    \caption{Code verantwoordelijk voor het uitvoeren van het script op srv1}
    \label{lst:bijlage-run-script}
\end{listing}

\begin{listing}
  \begin{minted}[linenos,tabsize=4,breaklines]{yaml}
- name: Download tarballs
  hosts:
    - servers
  gather_facts: no
  tasks:
    - name: Copy tarballs
      ansible.builtin.fetch:
        src: "/tmp/{{ inventory_hostname }}-configs.tar.gz"
        dest: "{{ lookup('ansible.builtin.env', 'HOME') }}"
        owner: "{{ lookup('ansible.builtin.env', 'USER') }}"
        group: "{{ lookup('ansible.builtin.env', 'USER') }}"
    - name: Extracting tarballs
      ansible.builtin.unarchive:
        src: "{{ lookup('ansible.builtin.env', 'HOME') }}/{{ inventory_hostname }}/tmp/{{ inventory_hostname }}-configs.tar.gz"
        dest: "{{ lookup('ansible.builtin.env', 'HOME') }}/"
      delegate_to: localhost
  \end{minted}
  \caption{Code verantwoordelijk voor het downloaden en uitpakken van de tarballs}
  \label{lst:bijlage-tarballs}
\end{listing}
