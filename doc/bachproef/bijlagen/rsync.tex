%%=============================================================================
%% Symlink kopïeren met rsync
%%=============================================================================

\chapter{\IfLanguageName{dutch}{Oplossing: Symlinks kopi\"{e}ren met rsync}{Solution: Copying symlinks with rsync}}
\label{ch:bijlage_symlink_kopieren_met_rsync}

Deze bijlage presenteert een alternatieve oplossing voor het kopi\"eren van symlinks, een beperking van de huidige implementatie van het script.
In de diff van~\ref{lst:bijlage-rsync-copy-alternative-diff} is te zien dat we \texttt{cp} vervangen door \texttt{rsync}, gevolgd door enkele opties die ervoor zorgen dat de symlink wordt gevolgd en het effectieve bestand wordt gekopieerd.
Een korte uitleg van de gebruikte opties:

\begin{itemize}
  \item \texttt{--force}: Overschrijf bestaande bestanden.
  \item \texttt{--archive}: Archiveer bestanden en directories recursief.
  \item \texttt{--recursive}: Kopieer directories recursief.
  \item \texttt{--copy-links}: Kopieer symlinks als een bestand en volg de link naar het effectieve bestand.
\end{itemize}

Echter, deze operatie zal falen op Debian-systemen, omdat enkele symlinks in\\ \texttt{/etc/systemd/system/} verwijzen naar bestanden die niet bestaan.
Om dit probleem op te lossen, is \texttt{|| :} toegevoegd aan het einde van het commando.
Dit zorgt ervoor dat de operatie nooit faalt, zelfs als er een fout optreedt.

Echter betekent dit ook dat andere fouten niet correct zullen worden afgehandeld, wat de robuustheid van het script kan verminderen.
Dit is dan ook de redenen waarom deze oplossing niet werd ge\"implementeerd in de uiteindelijke versie van het script.

Uiteindelijk moeten we ook de playbook aanpassen zodat het script op elk systeem wordt uitgevoerd.
De benodigde aanpassing omvat het toevoegen van \texttt{rsync} aan de lijst van vereiste pakketten, zoals weergegeven in~\ref{lst:bijlage-rsync-playbook-diff}.

\begin{listing}
  \begin{minted}[linenos,tabsize=4,breaklines]{diff}
diff --git a/src/tool/confiscan.sh b/src/tool/confiscan.sh
index 231954e..0e5acc8 100755
--- a/src/tool/confiscan.sh
+++ b/src/tool/confiscan.sh
@@ -427,7 +427,7 @@ for c in "${APP_CONFIGS[@]}"; do
         error "${c} no such file or directory." 2

     mkdir -p "${output_dir}/$(dirname "${c}")"
-    cp -r "${c}" "${output_dir}/$(dirname "${c}")"
+    rsync --force --archive --recursive --copy-links "${c}" "${output_dir}/$(dirname "${c}")" || :
 done

 # File Integrity Check of original files, excluding ./original.sha256
  \end{minted}
  \caption{Alternatieve aanpak om symlinks ook te kopi\"{e}ren.}
  \label{lst:bijlage-rsync-copy-alternative-diff}
\end{listing}

\begin{listing}
  \begin{minted}[linenos,tabsize=4,breaklines]{diff}
diff --git a/src/poc/ansible/poc.yml b/src/poc/ansible/poc.yml
index 1010ed1..24ffd3e 100644
--- a/src/poc/ansible/poc.yml
+++ b/src/poc/ansible/poc.yml
@@ -8,6 +8,7 @@
         name:
           - net-tools
           - dmidecode
+          - rsync
         state: latest
  \end{minted}
  \caption{Aanpassing van de playbook om \texttt{rsync} te installeren.}
  \label{lst:bijlage-rsync-playbook-diff}
\end{listing}
