\chapter{\IfLanguageName{dutch}{Stand van zaken}{State of the art}}%
\label{ch:stand-van-zaken}

% Tip: Begin elk hoofdstuk met een paragraaf inleiding die beschrijft hoe
% dit hoofdstuk past binnen het geheel van de bachelorproef. Geef in het
% bijzonder aan wat de link is met het vorige en volgende hoofdstuk.

% Pas na deze inleidende paragraaf komt de eerste sectiehoofding.

\section{Network and Information Security 2}%
\label{sec:nis2}

\section{Huidige aanpak van asset management}%
\label{sec:huidige_aanpak_van_asset_management}

\subsection{Infrastructure as Code}%
\label{sub:ias}

\subsection{Data Center Infrastructure Management}
\label{sub:dcim}

\subsection{IP Address Management}
\label{sub:ipam}

\subsection{Snipe-IT}
\label{sub:snipe-it}

Snip-IT is een open-source webapplicatie ontwikkeld door Grokability sinds 2013~\autocite{snipe-it-introduction}, die gericht is op IT-assetmanagement.
Het idee achter Snipe-IT komt voort uit de vroegere aanpak van het bedrijf, toen het voornamelijk nog gebruik maakte van spreadsheets om hun bedrijfsmiddelen te inventariseren.
Het doel was om een applicatie te ontwikkelen die deze taak op een meer georganiseerde en effici\"ente manier kon uitvoeren.

\begin{figure}[h!]
    \includegraphics[width=\textwidth]
    {./graphics/snipe-dashboard.png}
    \caption{\label{fig:snipe-it-dashboard}Snip-IT dashboard.}
\end{figure}

De software biedt een gebruiksvriendelijke webinterface (\ref{fig:snipe-it-dashboard}) waarmee bedrijfsmiddelen, licenties, garanties en meer gemakkelijk kunnen worden beheerd.
Wanneer men kiest voor de self-hosted optie is de software gratis beschikbaar, terwijl ook verschillende cloud-gebaseerde optie beschikbaar zijn tegen een jaarlijkse bijdrage.
Deze prijs varieert afhankelijk van de nodige features en support.
Alle code en services gerelateerd aan Snipe-IT zijn vrij beschikbaar op GitHub~\autocite{snipe-it-github}.

Enkele van de belangrijkste functies van Snipe-IT zijn~\autocite{snipe-it-features}, maar zijn niet beperkt tot:
\begin{itemize}
    \item Gemakkelijk zien welke assets zijn toegewezen, aan wie, en hun fysieke locatie
    \item In één klik inchecken
    \item Assetmodellen waarmee je gemeenschappelijke functies kunt groeperen
    \item Vereisen van Gebruikersacceptatie (Eindgebruikers EULA's/Gebruiksvoorwaarden) bij Uitchecken
    \item E-mailmeldingen voor het verlopen van garanties en licenties
    \item Integratie met de meeste handheld barcode scanners en QR-codelezer apps
    \item Snelle en eenvoudige asset-audit
    \item Voeg je eigen aangepaste velden toe voor extra assetattributen
    \item Assets gemakkelijk importeren en exporteren
    \item Genereer QR-code labels voor eenvoudige mobiele toegang en labeling
    \item Assets gemarkeerd als aanvraagbaar kunnen worden aangevraagd door een gebruiker
    \item Assets behouden een volledige geschiedenis inclusief uitchecken, inchecken en onderhoud
    \item Optionele digitale handtekeningen bij assetacceptatie
\end{itemize}

\subsection{Nagios}
\label{sub:nagios}

\subsection{Lansweeper}
\label{sub:lansweeper}

Lansweeper, opgericht in 2004, is een Belgisch commercieel IT discovery \& inventory platform~\autocite{lansweeper-history}.
Het is een veelgebruikte tool voor het scannen, bijhouden en beheren van IT-assets binnen een organisatie.
Met functionaliteiten voor zowel hardware- als software-inventarisatie, biedt Lansweeper gebruikers een uitgebreid overzicht van alle IT-assets in hun netwerk.

Het grote voordeel van Lansweeper ten opzichte van andere tools, zoals Snipe-IT, is dat het een volledig geautomatiseerde oplossing biedt~\autocite{lansweeper-features}.
Dit betekent dat het platform in staat is om automatisch alle IT-assets binnen een netwerk te detecteren, zonder handmatige configuratie of de noodzaak om op elk systeem een agent te installeren~\autocite{lansweeper-getting-started}.
Dit gebeurt met behulp van verschillende protocollen, waaronder SNMP.
De gedetecteerde informatie wordt vervolgens opgeslagen in een centrale database en is toegankelijk via een gebruiksvriendelijke webinterface.

\begin{figure}[h!]
    \includegraphics[width=\textwidth]
    {./graphics/lansweeper-dashboard.png}
    \caption{\label{fig:lansweeper-dashboard}Lansweeper dashboard.}
\end{figure}

Gebruikers kunnen aangepaste rapporten genereren over verschillende aspecten van hun IT-infrastructuur, zoals hardwareconfiguraties, softwarelicenties, patchniveaus en meer.
Deze rapporten zijn bruikbaar voor audits, nalevingscontroles en capaciteitsplanning.
