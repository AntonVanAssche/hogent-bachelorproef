%%=============================================================================
%% Voorwoord
%%=============================================================================

\chapter*{\IfLanguageName{dutch}{Woord vooraf}{Preface}}%
\label{ch:voorwoord}

%% TODO:
%% Het voorwoord is het enige deel van de bachelorproef waar je vanuit je
%% eigen standpunt (``ik-vorm'') mag schrijven. Je kan hier bv. motiveren
%% waarom jij het onderwerp wil bespreken.
%% Vergeet ook niet te bedanken wie je geholpen/gesteund/... heeft

Allereerst wil ik mijn waardering uitspreken voor Bert Deferme, mijn co-promotor tijdens het realiseren van deze bachelorproef.
Zijn deskundige begeleiding heeft een cruciale rol gespeeld bij het tot stand brengen van dit werk, met name bij de uitwerking en ontwikkeling van het Bash-script, en bij de daaropvolgende code reviews.
Zijn expertise en toewijding hebben mijn vaardigheden aangescherpt en mijn begrip verdiept.
Ik ben hem bijzonder dankbaar voor zijn voortdurende beschikbaarheid en steun wanneer ik zijn hulp nodig had.

Verder wil ik mijn dank uitspreken aan mijn promotor, de heer Thomas Clauwaert, voor het initi\"eren van het oorspronkelijke idee voor deze bachelorproef en voor het toewijzen ervan aan mij.
Zijn begeleiding en waardevolle feedback tijdens het schrijven van deze bachelorproef hebben enorm bijgedragen aan de kwaliteit en diepgang ervan.

Ook wil ik mijn dank uitspreken aan mijn stageplaats, die zo vriendelijk waren om de toestemming te geven dat ik enkele schermafbeeldingen van hun Puppetboard en Red Hat Satellite mocht gebruiken in dit bachelorproef.

Ik heb dit onderwerp gekozen omdat het de mogelijkheid bood om twee van mijn passies te combineren: Bash en het automatiseren van Linux-systemen.
Beide hebben al enkele jaren mijn aandacht getrokken, waardoor het voor mij vanzelfsprekend was om dit onderwerp op mij te nemen.

Tot slot wil ik een speciaal woord van dank richten aan mijn familie voor hun steun en aanmoediging gedurende het schrijven van deze bachelorproef en de vele jaren van studie aan de Hogeschool Gent.

Dendermonde, 7 mei 2024

Anton Van Assche
