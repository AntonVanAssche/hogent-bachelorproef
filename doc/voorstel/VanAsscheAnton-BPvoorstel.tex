%==============================================================================
% Sjabloon onderzoeksvoorstel bachproef
%==============================================================================
% Gebaseerd op document class `hogent-article'
% zie <https://github.com/HoGentTIN/latex-hogent-article>

% Voor een voorstel in het Engels: voeg de documentclass-optie [english] toe.
% Let op: kan enkel na toestemming van de bachelorproefcoördinator!
\documentclass{hogent-article}

% Invoegen bibliografiebestand
\addbibresource{voorstel.bib}

% Informatie over de opleiding, het vak en soort opdracht
\studyprogramme{Professionele bachelor toegepaste informatica}
\course{Bachelorproef}
\assignmenttype{Onderzoeksvoorstel}
% Voor een voorstel in het Engels, haal de volgende 3 regels uit commentaar
% \studyprogramme{Bachelor of applied information technology}
% \course{Bachelor thesis}
% \assignmenttype{Research proposal}

\academicyear{2023-2024}
\title{Automatisering van Linux systeem inventarisatie: Onderzoek en Proof-of-Concept}
\author{Anton Van Assche}
\email{anton.vanassche@student.hogent.be}
\supervisor[Co-promotor]{T. Clauwaert (HoGent, \href{mailto:thomas.clauwaert@hogent.be}{thomas.clauwaert@hogent.be})}

\specialisation{System \& Network Administrator}
\keywords{Cybersecurity, Automatisatie, Linux}

\begin{document}

\begin{abstract}
    Op 6 januari 2023 werd de Network and Information Security (NIS)-richtlijn opgevolgd door de nieuwe versie, NIS2, die aanzienlijke nieuwe voorschriften introduceerde op het gebied van cybersecurity.
    Deze update breidde ook het aantal sectoren uit dat aan deze richtlijnen moet voldoen.
    Lidstaten van de Europese Unie hebben tot oktober 2024 de tijd om deze richtlijnen in hun nationale wetgeving te implementeren, wat vooral voor KMO's een uitdaging kan zijn vanwege beperkte middelen.

    Deze bachelorproef richt zich specifiek op het ondersteunen van KMO's bij het voldoen aan de ``inventory of assets'' vereiste van deze richtlijnen door het ontwikkelen van een tool voor het inventariseren van Linux-systemen.
    De studie omvat een literatuurstudie over het belang van configuratie-inventarissen en risico-analyse, gevolgd door de ontwikkeling van een Bash-script als tool.
    Een Proof-of-Concept wordt uitgevoerd op vijf Linux-servers met specifieke use cases, waarbij de inventarisatie resultaten worden geanalyseerd.
    De verwachte resultaten omvatten een effectieve en nauwkeurige inventaris, wat de potentie benadrukt om het incident response proces te versterken en KMO's te ondersteunen bij het naleven van de NIS2-richtlijnen met betrekking tot ``inventory of assets''.
    Deze bachelorproef richt zich specifiek op het ondersteunen van KMO's bij het voldoen aan de ``inventory of assets'' vereiste van deze richtlijnen door het ontwikkelen van een tool voor het inventariseren van Linux-systemen.
\end{abstract}

\tableofcontents

% De hoofdtekst van het voorstel zit in een apart bestand, zodat het makkelijk
% kan opgenomen worden in de bijlagen van de bachelorproef zelf.
%---------- Inleiding ---------------------------------------------------------

\section{Introductie}%
\label{sec:introductie}

In het begin van 2023, op 6 januari, werd de NIS-richtlijn, wat staat voor "Network and Information Security", opgevolgd door zijn nieuwe versie, NIS2.
Deze update bracht verschillende nieuwe voorschriften met zich mee op het gebied van cyberbeveiliging, en het aantal sectoren dat aan deze richtlijnen moet voldoen, werd aanzienlijk uitgebreid.

De lidstaten van de Europese Unie hebben tot oktober 2024~\autocite{NIS2Directive2022} de tijd om deze nieuwe richtlijnen in hun nationale wetgeving te implementeren.
Dit geeft bedrijven een beperkte periode om hun processen aan te passen aan deze nieuwe wetgeving, een uitdaging die vooral voor KMO's problematisch kan zijn, gezien hun vaak beperkte middelen.

Een van de essenti\"ele vereisten van deze richtlijnen is dat bedrijven verplicht zijn om een ``inventory of assets'' op te stellen voor alle systemen en operaties die ze gebruiken.
Deze bachelorproef richt zich specifiek op het bieden van een oplossing voor KMO's, die nu wel moeten voldoen aan deze richtlijnen, voor het inventariseren van Linux-systemen, waarbij de focus ligt op het vergemakkelijken van het naleven van deze nieuwe regelgeving.

Om dit te bereiken, zal een script of tool worden ontwikkeld die automatisch op elke Linux-server kan worden uitgevoerd.
Deze tool biedt vervolgens een gedetailleerde inventaris van het systeem en de configuratie van verschillende applicaties.
Het uiteindelijke doel is om deze tool uit te voeren op vijf verschillende servers met specifieke use cases.
Op basis hiervan zal worden geconcludeerd of het implementeren van deze tool nuttig is voor KMO's.

%---------- Stand van zaken ---------------------------------------------------

\section{Literatuurstudie}%
\label{sec:litaratuurstudie}

Volgens een studie uit 2022~\autocite{Kotenko2022} is het opstellen van een configuratie inventaris essentieel voor het beoordelen van cybersecurityrisico's.
Aan de hand van een inventaris kan men potenti\"ele kwetsbaarheden gaan identificeren en de passende maatregelen gaan treffen.
Ook wordt er benadrukt dat het inventaris kan gebruikt worden om mogelijke cyberaanvalspaden te identificeren en het voorkomen van verliezen door cyberaanvallen.
Het regelmatig herevalu\"eren van het inventaris wordt benadrukt als een proactieve benadering om potenti\"ele incidenten te voorkomen.
Dit benadrukt het belang van een nauwkeurige inventarisatie voor het implementeren van geautomatiseerde detectie- en bewakingsmaatregelen.

In een boek geschreven door Antal~\autocite{Antal2010} worden verschillende systeemeigenschappen genoemd die essentieel zijn voor een inventaris.
Hoewel het boek niet specifiek gericht is op Linux-systemen, biedt het een algemeen overzicht van cruciale aspecten van een inventaris.
Hierbij wordt de configuratie van het besturingssysteem benadrukt, inclusief details zoals de kernel- en besturingssysteemversie.
Verder wordt het belang van het aantal geïnstalleerde applicaties en hun configuratie benoemd als significant in het inventarisatieproces.
Daarnaast wordt de netwerkconfiguratie, waaronder IP-adressen en firewallconfiguratie, als essentieel beschouwd.

%---------- Methodologie ------------------------------------------------------
\section{Methodologie}%
\label{sec:methodologie}

In deze studie wordt onderzoek gedaan naar de bijdrage van een configuratie-inventaris van Linux-systemen aan het initi\"ele proces van incident response bij een cyberaanval.
Het onderzoek zal bestaan uit zes fasen, zoals weergegeven in figuur~\ref{fig:flowchart}.

\subsection{Fase 1: Literatuurstudie}%
\label{sub:literatuurstudie}

De eerste fase omvat een grondige literatuurstudie die zich richt op het vergelijken van de impact van cyberbeveiligingsincidenten tussen bedrijven met en zonder een uitgebreide inventaris.
Deze fase duurt 2 weken in beslag nemen, en resulteert in een literatuurstudie die niet alleen de impact belicht, maar ook eigenschappen van Linux-systemen identificeert die relevant zijn voor het inventaris.

\subsection{Fase 2: Risico-analyse}%
\label{sub:risico_analyse}

De tweede fase richt zich op een risico-analyse, waarbij de ge\"identificeerde Linux-systeem\\eigenschappen binnen het inventaris kritisch worden ge\"evalueerd.
Via een diepgaande risico-analyse worden de impact van deze eigenschappen op het incidentresponseproces en hun bruikbaarheid beoordeeld.
Ook deze fase zal 2 weken in beslag nemen, en resulteert in een lijst van eigenschappen die een positieve bijdrage leveren aan het incidentresponseproces en een lijst van eigenschappen die een negatieve impact hebben.

\subsection{Fase 3: Ontwikkeling van de tool}%
\label{sub:ontwikkeling_van_de_tool}

De derde fase omvat de ontwikkeling van de tool, een Bash-script dat de eigenschappen van fase 2 gebruikt om een inventaris van het Linux-systeem op te stellen.
Het resultaat is een werkend Bash-script dat ongeveer 3 weken in beslag neemt.

\subsection{Fase 4: Proof-of-Concept}%
\label{sub:proof_of_concept}

Na de ontwikkelingsfase volgt de Proof-of-\\Concept, waarbij een omgeving wordt opgezet met 5 Linux-servers met elk hun specifieke taak.
Deze fase zal ongeveer 1 week duren en omvat het testen van het script op de opgezette omgeving, resulterend in een inventaris van de omgeving.

\subsection{Fase 5: Analyse van de resultaten}%
\label{sub:analyse_van_de_resultaten}

Na het uitvoeren van het Proof-of-\\Concept worden de resultaten geanalyseerd.
Deze analyse vormt de basis voor de conclusie, waarbij de focus ligt op de kwaliteit en bruikbaarheid van de inventaris.
Beoordelingscriteria omvatten de volledigheid van de inventaris, de aanwezigheid van fouten en de impact op het incidentresponseproces.
Deze fase duurt 1 week.

\subsection{Fase 6: Afronding van de scriptie}%
\label{sub:afronding_van_de_scriptie}

De afsluitende fase richt zich op de voltooiing van de paper.
Ontbrekende hoofdstukken worden aangevuld, en de tekst wordt nauwkeurig nagelezen.
Het resultaat is een voltooide scriptie die voldoet aan alle vereiste normen en essenti\"ele hoofdstukken omvat.
Deze fase neemt 2 weken in beslag.

\begin{figure}[h!]
    \includegraphics[width=.49\textwidth]
    {graphics/methodologie_flowchart.png}
    \caption{\label{fig:flowchart}Flowchart fasen methodologie}
\end{figure}

%---------- Verwachte resultaten ----------------------------------------------
\section{Verwacht resultaat, conclusie}%
\label{sec:verwachte_resultaten}

De literatuurstudie heeft duidelijk de voordelen van een uitgebreide inventaris aangetoond bij het incident response proces na een cyberaanval.
De risico-analyse heeft kritieke eigenschappen van Linux-systemen ge\"identificeerd voor inventarisatie, waarbij een evenwicht is gezocht tussen positieve bijdragen aan het incident response proces en mogelijke negatieve effecten.

De ontwikkelde tool biedt een praktische oplossing voor het inventariseren van Linux-systemen te automatiseren.
Tijdens de Proof-of-Concept fase is aangetoond dat het script effectief is.
De inventarisatie was volledig en nauwkeurig, wat de potentie benadrukt om het incident response proces te versterken.



\printbibliography[heading=bibintoc]

\end{document}
